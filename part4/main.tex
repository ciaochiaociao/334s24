\documentclass[11pt]{article}

\usepackage{hyperref}
\usepackage{cleveref}

% double spacing
\usepackage{setspace}
\doublespacing

\begin{document}
% ELEC\_ENG 334: Fundamentals of Blockchains and Decentralization - 
\title{\vspace{-5cm}\Large Fundamentals of Blockchains and Decentralization (4)}
\author{Chiao-Wei Hsu}
\date{May 12, 2024}
\maketitle

% Please submit a report in pdf. Please use double spacing between paragraphs and use 11 pt font size. Also please keep it within one or two pages.

% In the first paragraph, please state clearly your experiment settings. It should include the difficulty, the lambda parameter, and the duration of your experiment. Also, state the size of blocks. If size varies, report the average size.

We set the difficulty to 0x00FF... and the lambda parameter to 10,000 microseconds. We run the miner for 20 seconds. The results are shown in \Cref{tab:blocks}. The size of blocks is 124 btyes.


% The second paragraph is a table of the number of blocks. It should look like the following. The table may come with a one-sentence caption.

% |Process|1  |2  |3  |
% |---|---|---|---|
% |# Blocks mined|   |   |   |
% |# Blocks in blockchain|   |   |   |
\begin{table}[hb]
    \centering
    \begin{tabular}{|c|c|c|c|}
        \hline
        Process & 1 & 2 & 3 \\
        \hline
        \# Blocks mined & 54 & 65 & 56 \\
        \# Blocks in blockchain & 176 & 176 & 176 \\
        \hline
    \end{tabular}
    \caption{The number of blocks mined and the number of blocks in the blockchain. The difficulty is 0x00FF..., the lambda parameter is 10,000 microseconds, and the duration of the experiment is 167 seconds.}
    \label{tab:blocks}
\end{table}

% The third paragraph should give a 2-sentence analysis of the results in the table.
% - Firstly check whether the number of blocks in blockchain is consistent among the 3 processes, if not, try to explain.
% - Secondly, check whether the sum of blocks mined by 3 processes equals the number of blocks in blockchain, if not, try to explain.

The number of blocks in the blockchain is consistent among the three processes, 176. The sum of blocks mined by the three processes, 175, is 1 less than the number of blocks in the blockchain because the genesis block is not counted in the number of blocks mined.

% The fourth paragraph is a table of block delay. It should look like the following. The table may come with a one-sentence caption.

% |Process|1  |2  |3  |
% |---|---|---|---|
% |Average of Block Delay|   |   |   |

\Cref{tab:delay} shows the average block delay in seconds.

\begin{table}[hb]
    \centering
    \begin{tabular}{|c|c|c|c|}
        \hline
        Process & 1 & 2 & 3 \\
        \hline
        Average of Block Delay (ms) & 12.01 & 1.95 & 10.52 \\
        \hline
    \end{tabular}
    \caption{The average block delay in seconds. The difficulty is 0x00FF..., the lambda parameter is 10,000 microseconds, and the duration of the experiment is 167 seconds.}
    \label{tab:delay}
\end{table}

% The fifth paragraph should give a 3-sentence analysis of the results in the table.
% - Firstly check whether the block delay value is reasonable, if not, try to explain. Here reasonable means that it should not be more than several seconds, since the communication between processes is on one machine, and the block size is not large.
% - Secondly, check whether the delay in processes 1 and 3 is larger than that in process 2. And explain it by considering the topology (hint: no connection between process 3 and process 1).
% - Thirdly, in processes 1 and 3, observe whether the delay values have two distinguishable clusters. And explain it by considering the different routes of blocks sent by process 2 and process 1/3.

The block delay values are reasonable, as they are all only a few milliseconds. The delay in processes 1 and 3 is larger than that in process 2 because there is no direct connection between processes 1 and 3. In processes 1 and 3, the delay values does not have two distinguishable cluster because the routes of blocks sent by process 2 and process 1/3 are the same.
\end{document}
